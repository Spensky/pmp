
\paragraph{}
\begin{itemize}
\item
	\tab Posibilitatea proiectarii robotilor deschide multe orizonturi in ceea ce o sa urmeze in lumea tehnologiei. Ne folosim de ei zilnic, in diferite forme, fiind implicati in prezent  in multe activitati sociale, de siguranta si intretinere si se transforma, ironic, dintr-o solutie pentru rezolvarea necesitatilor fiecaruia, intr-o noua necesitate.\\
	\tab Robotul realizat si prezentat in cadrul acestui proiect scoate in evidenta o modalitate rapida si eficienta da a studia si de a pune in practa intr-un mod usor lucrurile invatate in domeniul ingineriei si dezvoltarii software. \\
	\tab De asemenea, exista posibilitatea extinderii rapide a functionalitatilor robotului. In prezent exista nenumarati senzori, placi de dezvoltare sau diferite componente care o data folosite in proiectarea robotului, ii permit acestuia sa realizeze sau  sa rezolve anumite sarcini din ce in ce mai complicate.\\
	\tab Desi acest proiect sublineaza doar o parte infima din utilizarea robotilor in lumea reala, cu ajutorul resurselor rezultatul ar putea aduce beneficii semnificative pentru o multime de oameni si diferite sectoare ale lumii.\\
	\tab Totodata, alte functionalitati interesante ce ar putea fi abordate si dezolvate pe viitor sunt functionalitati precum comunicare intre roboti prin gesturi, urmarirea unui robot lider, utilizarea memoriei EEPROM pentru a memora un traseu parcurs anterior si multe altele.\\

\end{itemize}
