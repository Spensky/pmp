\newcommand\tab[1][1cm]{\hspace*{#1}}
\graphicspath{ {images/} }
\paragraph{Introducere}

\begin{itemize}
\item
	\tab Tema abordata are rolul de a trece prin toate etapele elementare a prelucrarii grafice, abordand fiecare laborator. S-a incercat aprofundarea,  cat si dezvoltarea ideiilor din laboratoare cu ajutorul resurselor auxiliare \footnote{Se vor cita ajutoarele pe parcursul acestui document, cat si linkurile unde e cazul}. \\
	\tab Fotorealism-ul reprezinta o categorie a picturii moderne care a luat nastere la sfarsitul secolului XX. Dupa cum spune si numele, fotorealismul reprezinta un mijloc de exprimare artistica ce presupune acumularea informatiilor prin fotografie si transpunerea acestora in pictura, astfel incat pictura realizata va parea ea insasi o fotografie.In cazul nostru, texturile, iluminarea, umbrele ,cat si alte tehnici avansate de prelucrare grafica, contribuie intr-un mod egal si important fotorealismului scenei evaluate.\\
\end{itemize}