
\begin{verbatim}[fontsize=\small]
\item{nod}
class Nod:

    def __init__(self, stare, actiune, cost, parent):
        self.stare = stare
        self.actiune = actiune
        self.cost = cost
        self.parent = parent

    def getStare(self):
        return self.stare

    def getActiune(self):
        return self.actiune

    def getCost(self):
        return self.cost

    def setParent(self, parent):
        self.parent = parent

    def setActiune(self, actiune):
        self.actiune = actiune

    def getParent(self):
        return self.parent

\item{dfs}

 nod = Nod(problem.getStartState(), 0, 0, None) 
#initializam nodul cu contructorul clasei
    my_stack = util.Stack() # initializam stacul
    visited = [] # lista pentru nodurile vizitate
    directii = [] # lista ce o vom returna la final cu directiile
    print nod.getStare()
    my_stack.push(nod) #adaugam nodul de inceput
    while not my_stack.isEmpty():
        nod = my_stack.pop() # scoatem cate un nod din stack
        if problem.isGoalState(nod.getStare()):
            # va intra numai daca nodul scos a ajuns la gol
            #si va parcurge de la nodul gol pana la primul 
	#nod cu ajutorul referintei la parinte
            #fiecare nod, adaugandu-i directiile in lista de directii
            while nod.getStare() != problem.getStartState():
                print nod.getStare()
                directii.insert(0, nod.getActiune())
                nod = nod.getParent()
            return directii
        #dupa ce am scos nodul din frontiera
        # noastra(stack) il adaugam la vizitate
        visited.append(nod.getStare())
        for vecini in problem.getSuccessors(nod.getStare()):
            #ii vom explora vecinii nodului curent, adaugandu-i in frontiera
            #nodul adaugat este scos primul la urmatoarea iteratie
            #il vom adauga numai daca nu a mai fost in visited sau frontiera
            nod2 = Nod(vecini[0], vecini[1], 0, nod)
            if nod2.getStare() not in visited:
                if nod2 not in my_stack.list:
                        my_stack.push(nod2)
            else:
            #daca nodul este in vizitate il cautam si ii reactualizam 
	 #parintele cat si actiunea
            #alfel la rezzultatul final ar sari peste unele noduri itermediare
	 # daca s-a ajuns la gol
            # A->G
            # \   |
            # B ->D
            #in exemplul de sus, dfs pune in vizited A,B,D, il gaseste pe G 
	 #si ii seteaza parintele cu A
            #el va returna A->G ca si soltuie, dar drumul corect parcurs 
	 #este A->B->D->G,
                for celula in my_stack.list:
                    if nod2.getStare() == celula.getStare():
                        celula.setParent(nod)
                        celula.setActiune(nod2.getActiune())

\item{bfs}

  nod = Nod(problem.getStartState(), 0, 0, None)
    my_queue = util.Queue()
    visited = []
    directii = []

    my_queue.push(nod) 
   # diferenta dintre bfs, si dfs la implementare  
   #este ca vom folosi o coada in loc de stiva
    while not my_queue.isEmpty():
        nod = my_queue.pop()
        if problem.isGoalState(nod.getStare()):
            while nod.getStare() != problem.getStartState():
                print nod.getStare(), problem.getStartState()
                directii.insert(0, nod.getActiune())
                nod = nod.parent
            return directii
        visited.append(nod.getStare())
        for vecini in problem.getSuccessors(nod.getStare()):
            nod2 = Nod(vecini[0], vecini[1], 0, nod)
            #se descopera un vecin si daca nu e in 
	  #vizited se va cauta sa nu fie in frontiera
            #apoi se va adauga, nu mai este nevoie de
	# reimprospatarea parintelui si a directiei
            #ca la dfs, deoarece algoritmul va vizita toate 
	#nodurile vecine a nodului curent, iar
            #mai apoi va trece la nivelul inferior
            if vecini[0] not in visited:
                gasit = 0
                for celula in my_queue.list:
                    if celula.getStare() == nod2.getStare():
                        gasit = 1
                        break
                if not gasit:
                    my_queue.push(nod2)

\item{ucs}

 nod = Nod(problem.getStartState(), 0, 0, None)
    my_priority_queue = util.PriorityQueue()
    visited = []
    directii = []
    my_priority_queue.push(nod, nod.getCost()) 
# vom folosi o coada de prioritati pentru a arata care drumuri sunt
                                                #mai prioritare, mai ieftine

    while not my_priority_queue.isEmpty():
        nod = my_priority_queue.pop()
        if problem.isGoalState(nod.getStare()):
            while nod.getStare() != problem.getStartState():
                directii.insert(0, nod.getActiune())
                nod = nod.parent
            return directii
        if nod.getStare() not in visited:
            visited.append(nod.getStare())
            for vecini in problem.getSuccessors(nod.getStare()):
                nod2 = Nod(vecini[0], vecini[1], nod.getCost() + vecini[2], nod)
                gasit = 0
                #daca copilul nu este in visited, il vom adauga,
	      # impreuna cu costul de a ajunge acolo
                #format din costul de pana acum + costul copilului
                if vecini[0] not in visited:
                    my_priority_queue.push(nod2, nod2.getCost())
                #daca se va gasi un traseu mai ieftin, nodului 
	     #curent ii se vor actualiza informatiile
                #plus costul corespunzator
                    if nod.getCost() < nod2.getCost():
                       for celula in my_priority_queue.heap:
                           if celula[2].getStare() == nod2.getStare():
                               gasit = 1
                       if gasit:
                           my_priority_queue.update(nod2, nod2.getCost())

\item{astar}

 my_priority_queue = util.PriorityQueue()
    contor = 0
    nod = Nod(problem.getStartState(), 0, 0, None)
    visited = []
    directii = []
    euristica = heuristic(nod.getStare(), problem)
 #se va adauga euristica, care cere ca si parametrii starea
 # si problema, euristica poate fi manhattan, euclidian ,ori 
# o euristica implementata de noi
    contor += 1
    print contor
    my_priority_queue.push(nod, nod.getCost() + euristica)
    while not my_priority_queue.isEmpty():
        nod = my_priority_queue.pop()
        if problem.isGoalState(nod.getStare()):
            while nod.getStare() != problem.getStartState():
                directii.insert(0, nod.getActiune())
                nod = nod.parent
            return directii
        if nod.getStare() not in visited:
            visited.append(nod.getStare())
            for vecini in problem.getSuccessors(nod.getStare()):
                nod2 = Nod(vecini[0], vecini[1], nod.getCost() + vecini[2], nod)
                euristica = heuristic(nod2.getStare(), problem)
                contor += 1
                print contor
                gasit = 0
                if nod2.getStare() not in visited:
                    #se va adauga la costul nodului curent si euristica
                    my_priority_queue.push(nod2, nod2.getCost() + euristica)
                    # daca se va gasi un traseu mai ieftin,
	          # nodului curent ii se vor actualiza informatiile
                    # plus costul corespunzator
                    if nod.getCost() < nod2.getCost():
                        for celula in my_priority_queue.heap:
                            if celula[2].getStare() == nod2.getStare():
                                gasit = 1
                                break
                        if gasit:
                            my_priority_queue.update(nod2, nod2.getCost() + euristica)

\item{corners}

 def __init__(self, startingGameState):
       # self.walls = startingGameState.getWalls()
        #self.startingPosition = startingGameState.getPacmanPosition()
        #top, right = self.walls.height-2, self.walls.width-2
       # self.corners = list(((1, 1), (1, top), (right, 1), (right, top)))
        #for corner in self.corners:
          #  if not startingGameState.hasFood(*corner):
            #    print 'Warning: no food in corner ' + str(corner)
       # self._expanded = 0 # DO NOT CHANGE; Number of search nodes expanded

        #verificam daca starea initiala nu este deja intr-un colt,
        #si daca este, stergem pozitia din lista de colturi
        if self.startingPosition in self.corners:
            self.corners.remove(self.startingPosition)
        # print self.startingPosition, self.corners[0], 
       #self.corners[1], self.corners[2], self.corners[3]

    def getStartState(self):
        #nu vom returna doar starea, dar si lista
       # de colturi pentru implementariile viitoare
        return self.startingPosition, self.corners

    def isGoalState(self, state):
        """
        Returns whether this search state is a goal state of the problem.
        """
        #vom verifica daca lista de colturi a starii nodului este nula
        #daca nu este, inseamna ca vor mai fi colturi in care pacmanul
        #trebuie sa le viziteze
        stare, corners = state
        if len(corners) > 0:
            return False
        else:
            return True

    def getSuccessors(self, state):
        """
        Returns successor states, the actions they require, and a cost of 1.

         As noted in search.py:
            For a given state, this should return a list of triples, (successor,
            action, stepCost), where 'successor' is a successor to the current
            state, 'action' is the action required to get there, and 'stepCost'
            is the incremental cost of expanding to that successor
        """

       # successors = []
        #for action in [Directions.NORTH, Directions.SOUTH, Directions.EAST, Directions.WEST]:
           # stare, corners = state
            #corners1 = list(corners)
            #x,y = stare
           # dx, dy = Actions.directionToVector(action)
            #nextx = int(x + dx)
            #nexty = int(y + dy)
            #hitsWall = self.walls[nextx][nexty]
            nextNod = (nextx, nexty)
            #verificam daca nu loveste un zid, iar mai apoi verificam
            #daca pozitia viitoare va fi un colt, daca da il vom sterge
            #din lista de colturi, iar mai apoi vom returna sucesorul
            #cu lista modificata, sau nu, sub forma de 
	#tupla(pozitie,lista de corner),actiune,cost)
            if not hitsWall:
                if nextNod in corners:
                    corners1.remove(nextNod)
                successors.append(((nextNod, corners1), action, 1))

        self._expanded += 1 # DO NOT CHANGE
        return successors

\item{Q6}

 if problem.isGoalState(state): 
#if the position is one of teh corners heuristic function
# should return 0 for consistency
         return 0

    visited = list(state[1]) # the list of remaining corners to visited of the node
    pozitie = state[0] # the position of the node
    i = 0
    costul_total = 0 # total cost...

    #we will find the minimum corner from the current position
    #using minimumCorner() function which will search throw all the
    #corners are calculate the aprox distance using manhattanDistance
    #then it will return that corner from which we will add it's manhattanDist
    #with the current position to the total cost of the heuristic
    # then we will remove the corner from the list and we will recall the minimumCorner
    #function with this corner and the remaining corners to find the 
     #closest corner to this one
    for i in range(0, len(visited)):
        index = minimumCorner(pozitie, visited)
        xy0 = int(pozitie[0] - index[0])
        xy1 = int(pozitie[1] - index[1])
        costul_total += abs(xy0) + abs(xy1)
        pozitie = index
        visited.remove(index)
    return costul_total #Heuristic resulted in expansion of 1186 nodes

    #return len(list(state[1])) -- 
   #simplest heuristic  path length: 106  Heuristic resulted in expansion of 1908 nodes

    #return util.manhattanDistance(pozitie,minimumCorner(pozitie,visited))
    # path length: 106  Heuristic resulted in expansion of 1475 nodes

def minimumCorner(pozitie, corner):
    minNod = None
    minDist = 9999999999
    i = 0
    while i < len(corner):
        #dist = util.manhattanDistance(pozitie, corner[i])
        # print "LUNG", len(corners)
        # print "Pozitie ", poz, "Coners ", corners[i]
        # (xy1[0] - xy2[0]) ** 2 + (xy1[1] - xy2[1]) ** 2 ) ** 0.5 # euclidian
        xy0 = int(pozitie[0] - corner[i][0]) #manhattan
        xy1 = int(pozitie[1] - corner[i][1])
        dist = abs(xy0) + abs(xy1)
        if minDist > dist:
            minDist = dist
            minNod = corner[i]
        i += 1
    return minNod

\item{Q7}

position, foodGrid = state
    "*** YOUR CODE HERE ***"
    food = list(foodGrid.asList())
    costul_total = 0
    pozitie = position
    i = 0
    if problem.isGoalState(state): # verify that the problem is goal state
        return 0
    while i < len(food):
        #print "AA"
        index = minimDistanceFood(pozitie, food)
        #print "Min food ", index
        xy0 = int(pozitie[0] - index[0])
        xy1 = int(pozitie[1] - index[1])
        costul_total += abs(xy0) + abs(xy1)
        #print costul_total
        pozitie = index
        food.remove(index)
        i += 1
    return costul_total #8881

    # a = util.manhattanDistance(pozitie, minimDistanceFood(pozitie, food))
    # if a < 3:
    #     costul_total = (a + len(food) )/2
    # return costul_total

    #return util.manhattanDistance(pozitie, minimDistanceFood(pozitie, food)) 13898
    #return len(food) 12517

def minimDistanceFood(pozitie, food):

    minNod = None
    minDist = 999999999
    i = 0
    while i < len(food):
        xy0 = int(pozitie[0] - food[i][0])
        xy1 = int(pozitie[1] - food[i][1])
        dist = abs(xy0) + abs(xy1)
        if minDist > dist:
            minDist = dist
            minNod = food[i]
        i += 1
    return minNod

\item{propositional logic and fol}

exists x tehnician(x) -> human(x).      %declare the existance of the jobs
exists x consultant(x) -> human(x).     % and each job practicant is also human 
exists x planner(x) -> human(x).
exists x engineer(x) -> human(x).
exists x nutritionist(x) -> human(x).

human(a).               %axioms declaring all five persons that are humans
human(b). 
human(c).
human(d).
human(e).

younger(e,c).  		% axioms declaring the known information regarding age 
younger(c,b). 		%comparison between the humans
younger(d,a).
younger(b,d).

%here we say that if humans are doing a profesion they can not do another
 %profesion in other words professions are unique to each humans

all x (tehnician(x) <-> -consultant(x) & -planner(x) & -enginner(x) & -nutritionist(x)).
all x (consultant(x) <-> -tehnician(x) & -planner(x) & -enginner(x) & -nutritionist(x)).
all x (planner(x) <-> -tehnician(x) & -consultant(x) & -enginner(x) & -nutritionist(x)).
all x (engineer(x) <-> -tehnician(x) & -planner(x) & -consultant(x) & -nutritionist(x)).
all x (nutritionist(x) <-> -tehnician(x) & -planner(x) & -enginner(x) & -consultant(x)).

%if human(x) is a profession than the others can not be the same profession aswell
% there are no 2 humans with the same profession (given in the text info)
consultant(a) <-> -consultant(b) & -consultant(c) & -consultant(d) & -consultant(e).
tehnician(c) <-> -tehnician(b) & -tehnician(a) & -tehnician(d) & -tehnician(e).
tehnician(d) <-> -tehnician(b) & -tehnician(c) & -tehnician(a) & -tehnician(e).
engineer(d) <-> -engineer(b) & -engineer(c) & -engineer(a) & -engineer(e).
planner(c) <-> -planner(b) & -planner(a) & -planner(d) & -planner(e).
planner(d) <-> -planner(b) & -planner(c) & -planner(a) & -planner(e).
nutritionist(e) <-> -nutritionist(b) & -nutritionist(c) & -nutritionist(d) & -nutritionist(a).

%we are saying that any person who is younger than someone else can not be a
 %consultant because consultant is the oldest of them all.(some helping hand
 %for mace4 to find the solution)
all x (all y (younger(x,y) -> -consultant(x))).
%we are saying that any 
%all x (-engineer(x) & -enginner(d) & -consultant(x) -> younger(x,b)).
%any human who is older than human(b) can be an engineer or an consultant
%(given to help mace4 find the solution).
all x (younger(b,x) & -engineer(d) -> engineer(x) | consultant(x)).
all x (younger(x,d) -> planner(x) | nutritionist(x) | tehnician(x)).
%any engineer is older than human(b) which also implies that b can not be 
%an engineer 
all x (engineer(x) & -engineer(b) -> younger(b,x)).
%any planner is younger than any tehnician(given in the text info)
all x (planner(x) & tehnician(y) -> younger(x,y)).
%any nutritionist is younger than any planner(given in the text info)
all x (nutritionist(x) & planner(y) -> younger(x,y)).

\item{arithmetic library}

list(distinct).
[consultant,tehnician,planner,engineer,nutritionist].
[a,b,c,d,e].
end_of_list.

x < y <-> younger(x,y).
x > y <-> older(x,y).

younger(tehnician,consultant) & 
younger(nutritionist,consultant) &
younger(planner,consultant) & 
younger(engineer,consultant).

older(engineer,tehnician).
older(tehnician,planner).
older(planner,nutritionist).

younger(e,c).
younger(c,b).
younger(d,a).
younger(b,d).

\end{verbatim}


